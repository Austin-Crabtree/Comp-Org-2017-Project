\message{ !name(report.tex)}\documentclass{llncs}
\setcounter{tocdepth}{5}

\setcounter{secnumdepth}{5}
\usepackage{xcolor}
\usepackage{graphicx}
\usepackage{listings}
\usepackage[utf8]{inputenc}
\lstset{language=Verilog,breaklines=true,basicstyle=\footnotesize, frame=single}
\usepackage{scrextend}
\usepackage{hyperref}
\hypersetup{
    colorlinks,
    citecolor=black,
    filecolor=black,
    linkcolor=violet,
    urlcolor=black
}

\def\cm#1#2{\list{}{\rightmargin#2\leftmargin#1}\item[]}
\let\endchangemargin=\endlist 

\begin{document}

\message{ !name(report.tex) !offset(35) }
\section{Design Methodology}
The MIPS instruction set is designed with the ideal of a simple instruction set that includes instructions that largely use the same set of
 logical operations to carry out their task.  Our design methodology stays true to that, especially with the small instruction set.  We implement
 all of the basic features of a MIPS processor.  This design includes an Instruction memory, Registers, Data memory, ALU operations etc...
 We also have implemented some features to deal with the opportunities for hazards to occur, such as: Data, Stuctural and Control.

\message{ !name(report.tex) !offset(150) }

\end{document}
